\chapter{Basics and Definitions}

The first thing is to install \aoe{}, for instance, via \steam{}\cite{Ozhara:2017}. After that, you need to identify \genie{} program: it is usually installed at \verb|C:\Program Files (x86)\Steam\steamapps\common\AoE2DE\Tools_Builds\AdvancedGenieEditor3.exe|. Data mods are discourages\cite{yorok:2019}, hence you should create modes by replacintg individual files and avoiding data modding whenever possible. We will refer to the folder \verb|C:\Program Files (x86)\Steam\steamapps\common\AoE2DE\| as \aoeexedir{}.

The mods you install from \aoe{} are available in a directory that usually is \verb|%HOMEPATH%\Games\Age of Empires 2 DE| (from here on dubbed \aoehomedir{}).
Within \aoehomedir{}, there should be a folder whose name is the user \textbf{steam id} (which is a big number) (from here on such a folder is dubbed as \aoeweirdnumberdir{})\cite{steamid:2019}. Within it, there is a folder called \verb|mods\subscribed| containing all the mods the user has subscribed to. Along side \dquote{subscribed} folder, there is a folder named \dquote{local} where local mods in development may be put (such a folder will be referred to as \aoehomelocaldir{}). The structure of \aoehomedir{} is shown in \lstref{verb1}.

\lstinputlisting[label=verb1,caption={Age of Empires home folder directory structure}, float, frame=ht]{src/codes/folder-structure.txt}

For example, \tblref{tbl:paths} shows all the involved paths for the modding.

\begin{table}[ht]
    \centering
    \begin{tabular}{ll}
        \toprule
        Path & Example \\
        \midrule
        \aoeexedir{}            & \code{C:\textbackslash{}Program Files (x86)\textbackslash{}Steam\textbackslash{}steamapps\textbackslash{}common\textbackslash{}AoE2DE\textbackslash{}} \\
        \aoehomedir{}           & \code{C:\textbackslash{}Users\textbackslash{}FooBar\textbackslash{}Games\textbackslash{}Age of Empires 2 DE} \\
        \aoeweirdnumberdir{}    & \code{C:\textbackslash{}Users\textbackslash{}FooBar\textbackslash{}Games\textbackslash{}Age of Empires 2 DE\textbackslash{}9823578647902347890} \\
        \bottomrule
    \end{tabular}
    \caption{Examples of important paths used in \aoe{}}
    \label{tbl:paths}
\end{table}

In the \dquote{mods} folder, there is a file called \code{mods-status.json} (an example of the content is shown in \lstref{verb2}).

\lstinputlisting[language=json,caption=Example of mods-status file,label=verb2, frame=ht]{src/codes/mods-status.json}

The json stored is a sequence of objects, each representing:

\begin{itemize}
    \item Checksum: MD5 of the mod?
    \item \textbf{Enable}: if \true{}, the mod is enabled, \false{} otherwise;
    \item Last Update: timestamp representing the number of milliseconds when the mod has been lastly updated;
    \item \textbf{Path}: path, relative to \aoeweirdnumberdir{} directory, where the specific mod is installed;
    \item \textbf{Priority}: priority used to load the mod;
    \item Publish Id: always 0?
    \item \textbf{Title}: name of the mod to show to the user;
    \item WorkShop Id: An id that uniquely identifies this mod on \url{www.ageofempires.com} site;
\end{itemize}


At high level \aoe{} install each mod by following \algref{alg:bootstrap}. 
After sorting out the enabled mods, the software \dquote{patches} the set of files specified by the mod path.
From the algorithm, it can be seen that the priority of each mod determine the order each mod is used. Priority may change the behavior is a mod $x$ relies on the installation of the mod $y$ (if $y.priority < x.priority$)\cite{Ozhara:2017}.

\begin{algorithm}
    \SetAlgoLined
    $mods \leftarrow$ Gather mods in \aoeweirdnumberdir{}\;
    $sortedmods \leftarrow$ Sort $mods$ by prioritzing mods with small priority\;
    \ForEach(mods subscribed){$m \in reversed(sortedmods)$}{
        \If{$\neg m.Enable$}{
            \textbf{continue}\;
        }
        Copy the files in \aoeweirdnumberdir{}\textbackslash{}mods\textbackslash{}$m.Path$ into \aoeexedir{}\;
    }
    Start \aoe{}\;
    \caption{\aoe{} mod boot strap algorithm}
    \label{alg:bootstrap}
\end{algorithm}